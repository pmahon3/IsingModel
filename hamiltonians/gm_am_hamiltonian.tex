
\documentclass{article}
\usepackage{amsmath}

\begin{document}

\title{Hamiltonian for the Ising Model with Fixed Sum of Spins}
\author{}
\date{}
\maketitle

\section*{Introduction}
This document derives the Hamiltonian for an Ising model where the spins \( s_i \) are constrained to sum to 1, and each spin takes values in the interval \([0,1]\). The Hamiltonian is defined to be minimized when the volume of the hypercube formed by the spins is maximized, and maximized when the volume is minimized.

\section*{Definitions}
Given a set of spins \( \{s_1, s_2, \ldots, s_N\} \) such that:
\[
\sum_{i=1}^N s_i = 1
\]
the arithmetic mean (AM) and geometric mean (GM) are given by:
\[
\text{AM} = \frac{1}{N}
\]
\[
\text{GM} = \left( \prod_{i=1}^N s_i \right)^{\frac{1}{N}}
\]

\section*{Hamiltonian as the Ratio of GM to AM}
We define the Hamiltonian \( H \) as the ratio of the geometric mean to the arithmetic mean. This ratio is:
\[
\frac{\text{GM}}{\text{AM}} = \left( \prod_{i=1}^N s_i \right)^{\frac{1}{N}} \cdot N
\]
Taking the natural logarithm of this ratio, we get:
\[
\ln\left( \frac{\text{GM}}{\text{AM}} \right) = \ln\left( \left( \prod_{i=1}^N s_i \right)^{\frac{1}{N}} \cdot N \right)
\]
This simplifies to:
\[
\ln\left( \frac{\text{GM}}{\text{AM}} \right) = \ln\left( \left( \prod_{i=1}^N s_i \right)^{\frac{1}{N}} \right) + \ln(N)
\]
Further simplification gives:
\[
\ln\left( \frac{\text{GM}}{\text{AM}} \right) = \frac{1}{N} \sum_{i=1}^N \ln(s_i) + \ln(N)
\]
Multiplying both sides by \( N \) to form the Hamiltonian \( H \):
\[
H = N \ln\left( \frac{\text{GM}}{\text{AM}} \right) = \sum_{i=1}^N \ln(s_i) + N \ln(N)
\]

\section*{Summary}
The Hamiltonian for the Ising model with the given constraints is:
\[
H = \sum_{i=1}^N \ln(s_i) + N \ln(N)
\]
This Hamiltonian is minimized when the geometric mean is maximized (i.e., when all spins are equal) and maximized when the geometric mean is minimized (i.e., when one spin is 1 and the rest are 0).

\end{document}
